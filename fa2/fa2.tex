%%%%%%%%%%%%%%%%%%%%%%%%%%%%%%%%%%%%%%%%%
% Beamer Presentation
% LaTeX Template
% Version 1.0 (10/11/12)
%
% This template has been downloaded from:
% http://www.LaTeXTemplates.com
%
% License:
% CC BY-NC-SA 3.0 (http://creativecommons.org/licenses/by-nc-sa/3.0/)
%
%%%%%%%%%%%%%%%%%%%%%%%%%%%%%%%%%%%%%%%%%

%----------------------------------------------------------------------------------------
%	PACKAGES AND THEMES
%----------------------------------------------------------------------------------------

\documentclass{beamer}

\mode<presentation> {

\usetheme{CambridgeUS}

}

\usepackage{graphicx} % Allows including images
\DeclareGraphicsExtensions{.png}
\usepackage{booktabs} % Allows the use of \toprule, \midrule and \bottomrule in tables
\usepackage[parfill]{parskip}

%----------------------------------------------------------------------------------------
%	TITLE PAGE
%----------------------------------------------------------------------------------------

\title[Sublinear Space Pattern Matching]{The Theory and Practice of Sublinear Space Pattern Matching} % The short title appears at the bottom of every slide, the full title is only on the title page

\author{Dominic Moylett} % Your name
\institute[University of Bristol] % Your institution as it will appear on the bottom of every slide, may be shorthand to save space
{
University of Bristol \\ % Your institution for the title page
\medskip
\textit{dominic.moylett.2011@my.bristol.ac.uk} % Your email address
}
\date{\today} % Date, can be changed to a custom date

\begin{document}

\begin{frame}
\titlepage % Print the title page as the first slide
\end{frame}

%----------------------------------------------------------------------------------------
%	PRESENTATION SLIDES
%----------------------------------------------------------------------------------------

%------------------------------------------------
\section{The data is getting bigger}
%------------------------------------------------

\begin{frame}
\frametitle{'Big Data'}
``I remember when this topic was called `Massive Data'. Now it has gone from `Massive Data' to `Big Data'.''
\textit{Graham Cormode}\footnote{\url{https://youtu.be/AXsBQBzKfYw?t=35s}}

The amount of data we are processing is getting bigger than we could've previously imagined.

We need algorithms that can handle these volumes of data in as little space as possible.
\end{frame}

%------------------------------------------------

\begin{frame}
\frametitle{What's the problem with big data?}
\begin{center}
\includegraphics[width=0.5\paperwidth]{moores_law}
\end{center}
"Sure, data's getting bigger, but so are our computers! Why can't we just throw more RAM at the problem?"
\end{frame}

%------------------------------------------------

\begin{frame}
\frametitle{May's Law}
``Software efficiency halves every 18 months, compensating for Moore's Law.''
\textit{David May}\footnote{\url{http://www.linux-mag.com/id/8422/}}

Just because the hardware is getting quicker doesn't mean we can utilise that speedup.

Particularly as we move from single to parallel computation.
\end{frame}

%------------------------------------------------

\begin{frame}
\frametitle{That data is growing too fast}
Between 1999 and 2015, the size of RAM on commercial machines increased roughly 100 times over.\footnote{Calculated from \url{http://www.jcmit.com/memoryprice.htm}}

In comparison, between 1999 and 2012, the amount of webpages indexed at Google increased 1000 times over.\footnote{\url{http://readwrite.com/2012/02/29/interview_changing_engines_mid-flight_qa_with_goog}}
\end{frame}

%------------------------------------------------

\begin{frame}
\frametitle{More memory isn't always the answer}
Some devices we don't want to keep adding more RAM to.

We want to save physical space in routers and mobile phones to keep them small.

But at the same time, we want them to process a lot of data.

There has to be a compromise.
\end{frame}

%------------------------------------------------
\section{Why pattern matching?}
%------------------------------------------------

\begin{frame}
\frametitle{Broad applications}
Pattern matching is a generic collection of problems that have a wide range of applications:

\begin{itemize}
    \item Bioinformatics
    \item Web searching
    \item Plagiarism checking
\end{itemize}
\end{frame}

%------------------------------------------------

\begin{frame}
\frametitle{Space constraints}
Historically, pattern matching has required linear space. As the pattern grows, your data structure grows at the same rate.

Intuitively, this makes sense. You want to check the text completely matches the pattern, so surely you need to store the whole pattern in order to do that?

However, Porat and Porat (2009) proved that with some chance of error, you can achieve pattern matching in logarithmic space, becoming the first to break this barrier.
\end{frame}

%------------------------------------------------

\begin{frame}
\frametitle{Other accomplishments}
Breslauer and Galil (2011) improved Porat and Porat's complexity from $O(n\log m)$ to $O(n)$ time.

Jalsenius, Porat and Sach (2012) created a solution to parameterised matching in sublinear space and $O(n)$ time.

Clifford, Fontaine, Porat and Sach (to be published) created a solution to pattern matching with multiple patterns in sublinear space and $O(n\log m)$ time.
\end{frame}

%------------------------------------------------

\begin{frame}
\frametitle{Some theory problems}
\textbf{Parameterised matching:} Current space lower bound is linear in the size of the alphabet. Can we bridge this gap?

\textbf{Dictionary matching:} Can we improve the solution to $O(n)$ time?

\textbf{Parameterised dictionary matching:} Currently no sublinear solution known.
\end{frame}

%------------------------------------------------

\begin{frame}
\frametitle{Pratical problems}
Very little research at practical performance of algorithms.

First implementation of Breslauer and Galil was last year.

How can we make these algorithms run better in practice?

Can we incorporate practical concepts such as parallelism or vectorisation?
\end{frame}

%------------------------------------------------

\begin{frame}
\frametitle{Work plan}
\textbf{Years 1-2:} Develop and improve current algorithms for pattern matching problems.

\textbf{Years 2-3:} Implement new and recent algorithms to see how they perform when compared to classic methods on real data.

\textbf{Years 3-4:} Discover practical bottlenecks and improve performance.

\textbf{Years 3-5:} Release implementations as practical tools for the open-source community.
\end{frame}

%------------------------------------------------
\section{The End}
%------------------------------------------------

\begin{frame}
\Huge{\centerline{Questions?}}
\end{frame}

%----------------------------------------------------------------------------------------

\end{document} 
