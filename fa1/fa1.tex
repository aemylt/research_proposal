%%%%%%%%%%%%%%%%%%%%%%%%%%%%%%%%%%%%%%%%%
% Beamer Presentation
% LaTeX Template
% Version 1.0 (10/11/12)
%
% This template has been downloaded from:
% http://www.LaTeXTemplates.com
%
% License:
% CC BY-NC-SA 3.0 (http://creativecommons.org/licenses/by-nc-sa/3.0/)
%
%%%%%%%%%%%%%%%%%%%%%%%%%%%%%%%%%%%%%%%%%

%----------------------------------------------------------------------------------------
%	PACKAGES AND THEMES
%----------------------------------------------------------------------------------------

\documentclass{beamer}

\mode<presentation> {

% The Beamer class comes with a number of default slide themes
% which change the colors and layouts of slides. Below this is a list
% of all the themes, uncomment each in turn to see what they look like.

%\usetheme{default}
%\usetheme{AnnArbor}
%\usetheme{Antibes}
%\usetheme{Bergen}
%\usetheme{Berkeley}
%\usetheme{Berlin}
%\usetheme{Boadilla}
\usetheme{CambridgeUS}
%\usetheme{Copenhagen}
%\usetheme{Darmstadt}
%\usetheme{Dresden}
%\usetheme{Frankfurt}
%\usetheme{Goettingen}
%\usetheme{Hannover}
%\usetheme{Ilmenau}
%\usetheme{JuanLesPins}
%\usetheme{Luebeck}
%\usetheme{Madrid}
%\usetheme{Malmoe}
%\usetheme{Marburg}
%\usetheme{Montpellier}
%\usetheme{PaloAlto}
%\usetheme{Pittsburgh}
%\usetheme{Rochester}
%\usetheme{Singapore}
%\usetheme{Szeged}
%\usetheme{Warsaw}

% As well as themes, the Beamer class has a number of color themes
% for any slide theme. Uncomment each of these in turn to see how it
% changes the colors of your current slide theme.

%\usecolortheme{albatross}
%\usecolortheme{beaver}
%\usecolortheme{beetle}
%\usecolortheme{crane}
%\usecolortheme{dolphin}
%\usecolortheme{dove}
%\usecolortheme{fly}
%\usecolortheme{lily}
%\usecolortheme{orchid}
%\usecolortheme{rose}
%\usecolortheme{seagull}
%\usecolortheme{seahorse}
%\usecolortheme{whale}
%\usecolortheme{wolverine}

%\setbeamertemplate{footline} % To remove the footer line in all slides uncomment this line
%\setbeamertemplate{footline}[page number] % To replace the footer line in all slides with a simple slide count uncomment this line

%\setbeamertemplate{navigation symbols}{} % To remove the navigation symbols from the bottom of all slides uncomment this line
}

\usepackage{graphicx} % Allows including images
\usepackage{booktabs} % Allows the use of \toprule, \midrule and \bottomrule in tables

%----------------------------------------------------------------------------------------
%	TITLE PAGE
%----------------------------------------------------------------------------------------

\title[Challenges of open innovation]{Challenges of open innovation: the paradox of firm investment in open-source software} % The short title appears at the bottom of every slide, the full title is only on the title page

\author{Dominic Moylett} % Your name
\institute[University of Bristol] % Your institution as it will appear on the bottom of every slide, may be shorthand to save space
{
University of Bristol \\ % Your institution for the title page
\medskip
\textit{dominic.moylett.2011@my.bristol.ac.uk} % Your email address
}
\date{\today} % Date, can be changed to a custom date

\begin{document}

\begin{frame}
\titlepage % Print the title page as the first slide
\end{frame}

%----------------------------------------------------------------------------------------
%	PRESENTATION SLIDES
%----------------------------------------------------------------------------------------

%------------------------------------------------
\section{Open Innovation and its Challenges}
%------------------------------------------------

\begin{frame}
\frametitle{From Central to Open Innovation}
Central innovation such as R\&D labs has fallen by the wayside over recent years. This is because of the difficulty in exploiting ideas from central innovation. A 'man of genius' may have smart ideas, but far too often these ideas cannot be commercialised. They end up 'sat on a shelf', waiting for further development.\\~\\

Open innovation goes beyond internal development by exploring opportunities for innovation, integrating these with internal development, and exploiting these opportunities. This change can yield new opportunities, but will also open up new challenges. Challenges which need to be resolved by different strategies.
\end{frame}

%------------------------------------------------

\begin{frame}
\frametitle{Open Innovation: Three Challenges}
\begin{itemize}
\item \textbf{Maximisation:} Open innovation offers more opportunities for how to generate profit, including commercialisation, but also pooling and licensing. How can firms pick the method that maximises return on investment?
\item \textbf{Incorporation:} Opportunities for innovation from external sources existing alone does not offer any benefit to the firm. The firm needs to find these opportunities and be able \& willing to integrate them into the main business.
\item \textbf{Motivation:} What benefit do firms stand to receive from releasing spillovers? Why would they spend money on R\&D if the result will be available to competitors?
\end{itemize}
\end{frame}

%------------------------------------------------
\section{Open Innovation Strategies}
%------------------------------------------------

\begin{frame}
\frametitle{Solutions in Open-Source Software}
West and Gallagher investigated this problem by interviewing a number of organisations that contribute to open-source projects. From this research, they grouped open innovation methods into four categories:
\begin{table}
\begin{tabular}{l l}
\toprule
\textbf{Structural} & \textbf{Product-centric}\\
\midrule
Pooled R\&D & Selling complements \\
Spinout & Donated complements \\
\bottomrule
\end{tabular}
\end{table}
\end{frame}

%------------------------------------------------
\subsection{Structural Methods}
%------------------------------------------------

\begin{frame}
\frametitle{Pooled R\&D}
Firms collaborate on innovations together and contribute to a shared resource of R\&D.\\~\\

Spillover benefits firms by establishing legitimacy in the community.\\~\\

Mozilla Firefox was originally released open-source due to competition from Microsoft's Internet Explorer. Companies such as IBM and HP contributed to the project to ensure that the workstations they sold could have a Unix-based web browser.\\~\\

NB: Firefox is atypical for Pooled R\&D. The open-source licence means that limiting access to internal innovations is difficult, and contributions can also come from outside the sponsoring companies.
\end{frame}

%------------------------------------------------

\begin{frame}
\frametitle{Spinouts}
Occur when a firm releases an innovation to the public as an open-source stand-alone project. This might just be done when products no longer generate revenue or meet strategy, such as with Xerox PARC. But it can also be done strategically to make the technology more valuable.\\~\\

One strategic spinout is Eclipse, developed initially by IBM. Java development tools were predicted to become a commodity and thus products like Eclipse were going to have little commercial value.\\~\\

Eclipse was released into the public domain to establish it as a \textit{de facto} standard. IBM was then able to sell complementary products such as WebSphere to make a return.\\~\\

Most often used for largest firms with the highest innovation budgets.
\end{frame}

%------------------------------------------------
\subsection{Product-Centric Methods}
%------------------------------------------------

\begin{frame}
\frametitle{Selling Complements}
Central product is based on an external innovation, yet complements are developed internally and commercialised.\\~\\

Apple started Darwin to contribute its changes to BSD Unix back to the open-source community. But all remaining code behind the OSX operating system remained private.\\~\\

Another common strategy is to open-source the product, but offer training and support services, as is the case with Red Hat Enterprise Linux.
\end{frame}

%------------------------------------------------

\begin{frame}
\frametitle{Donated Complements}
The central product is developed internally, but tools are made available externally to allow others to modify or expand on the innovation.\\~\\

Internal developers benefit from extended lifetime of their core innovation, and external developers gain recognition.\\~\\

Alongside the Half Life Games, Valve Corporation released Team Fortress Classic to promote their development kit. This lead to community mods such as Counter Strike and The Stanley Parable.\\~\\

Id Software released Quake publicly to encourage modifications to their game. Source code is licensed so that Id receives a portion of any sales made from mods.\\~\\

Typically sold to technically proficient audiences.
\end{frame}

%------------------------------------------------
\section{Further Challenges}
%------------------------------------------------

\begin{frame}
\frametitle{Further Challenges}
\begin{itemize}[<+->]
\item Open innovation can introduce higher coordination costs.
\item As innovations become more commercialised, the principles and ideals behind open-source could disappear.
\item Popularity for open-source could wade if projects fail against competitors such as Microsoft.
\item Risk of users (un)knowingly contributing proprietry to open-source projects due to accepting donations from unknown users.
\end{itemize}
\end{frame}

%------------------------------------------------
\section{Critique}
%------------------------------------------------

\begin{frame}
\frametitle{Figure}
Uncomment the code on this slide to include your own image from the same directory as the template .TeX file.
%\begin{figure}
%\includegraphics[width=0.8\linewidth]{test}
%\end{figure}
\end{frame}

%------------------------------------------------

\begin{frame}[fragile] % Need to use the fragile option when verbatim is used in the slide
\frametitle{Citation}
An example of the \verb|\cite| command to cite within the presentation:\\~

This statement requires citation \cite{p1}.
\end{frame}

%------------------------------------------------

\begin{frame}
\frametitle{References}
\footnotesize{
\begin{thebibliography}{99} % Beamer does not support BibTeX so references must be inserted manually as below
\bibitem[Smith, 2012]{p1} John Smith (2012)
\newblock Title of the publication
\newblock \emph{Journal Name} 12(3), 45 -- 678.
\end{thebibliography}
}
\end{frame}

%------------------------------------------------

\begin{frame}
\Huge{\centerline{Questions?}}
\end{frame}

%----------------------------------------------------------------------------------------

\end{document} 
