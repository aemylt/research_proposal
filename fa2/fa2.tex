%%%%%%%%%%%%%%%%%%%%%%%%%%%%%%%%%%%%%%%%%
% Beamer Presentation
% LaTeX Template
% Version 1.0 (10/11/12)
%
% This template has been downloaded from:
% http://www.LaTeXTemplates.com
%
% License:
% CC BY-NC-SA 3.0 (http://creativecommons.org/licenses/by-nc-sa/3.0/)
%
%%%%%%%%%%%%%%%%%%%%%%%%%%%%%%%%%%%%%%%%%

%----------------------------------------------------------------------------------------
%	PACKAGES AND THEMES
%----------------------------------------------------------------------------------------

\documentclass{beamer}

\mode<presentation> {

\usetheme{CambridgeUS}

}

\usepackage{graphicx} % Allows including images
\DeclareGraphicsExtensions{.png, .pdf}
\usepackage{booktabs} % Allows the use of \toprule, \midrule and \bottomrule in tables
\usepackage[parfill]{parskip}

%----------------------------------------------------------------------------------------
%	TITLE PAGE
%----------------------------------------------------------------------------------------

\title[Sublinear Space Pattern Matching]{The Theory and Practice of Sublinear Space Pattern Matching} % The short title appears at the bottom of every slide, the full title is only on the title page

\author{Dominic Moylett} % Your name
\institute[University of Bristol] % Your institution as it will appear on the bottom of every slide, may be shorthand to save space
{
University of Bristol \\ % Your institution for the title page
\medskip
\textit{dominic.moylett.2011@my.bristol.ac.uk} % Your email address
}
\date{\today} % Date, can be changed to a custom date

\begin{document}

\begin{frame}
\titlepage % Print the title page as the first slide
\end{frame}

%----------------------------------------------------------------------------------------
%	PRESENTATION SLIDES
%----------------------------------------------------------------------------------------

%------------------------------------------------
\section{The data is getting bigger}
%------------------------------------------------

\begin{frame}
\frametitle{'Big Data'}
``I remember when this topic was called `Massive Data'. Now it has gone from `Massive Data' to `Big Data'.''
\textit{Graham Cormode}\footnote{\url{https://youtu.be/AXsBQBzKfYw?t=35s}}
\end{frame}

%------------------------------------------------

\begin{frame}
\frametitle{What's the problem with big data?}
\begin{center}
\includegraphics[width=0.4\paperwidth]{moores_law}\footnote{\url{http://upload.wikimedia.org/wikipedia/commons/0/00/Transistor_Count_and_Moore\%27s_Law_-_2011.svg}}
\end{center}
"Sure, data's getting bigger, but so are our computers! Why can't we just throw more RAM at the problem?"
\end{frame}

%------------------------------------------------

\begin{frame}
\frametitle{May's Law}
``Software efficiency halves every 18 months, compensating for Moore's Law.''
\textit{David May}\footnote{\url{http://www.linux-mag.com/id/8422/}}
\end{frame}

%------------------------------------------------

\begin{frame}
\frametitle{Data is growing too fast}
Between 1999 and 2015, the size of RAM on commercial machines increased roughly 100 times over.\footnote{Calculated from \url{http://www.jcmit.com/memoryprice.htm}}

In comparison, between 1999 and 2012, the amount of webpages indexed at Google increased 1000 times over.\footnote{\url{http://readwrite.com/2012/02/29/interview_changing_engines_mid-flight_qa_with_goog}}
\end{frame}

%------------------------------------------------

\begin{frame}
\frametitle{More memory isn't always the answer}
Some devices we don't want to keep adding more RAM to.

We want to save physical space in routers and mobile phones to keep them small.

But at the same time, we want them to process a lot of data.

There has to be a compromise.
\end{frame}

%------------------------------------------------
\section{Why pattern matching?}
%------------------------------------------------

\begin{frame}
\frametitle{Broad applications}
Pattern matching is a generic collection of problems that have a wide range of applications:

\begin{itemize}
    \item Bioinformatics
    \item Web searching
    \item Plagiarism checking
\end{itemize}
\end{frame}

%------------------------------------------------

\begin{frame}
\frametitle{We don't need to store the whole text}
The data streaming model is a model of computation where we don't read in the whole input at once.

Instead, we read in portions of the input.

Pattern matching easily applies to this. Some algorithms such as Knuth-Morris-Pratt only read each character in the text once, so they can easily be streamed instead.

The streaming model is also more relistic to some low-memory applications, such as routers.
\end{frame}

%------------------------------------------------

\begin{frame}
\frametitle{Space constraints}
Historically, pattern matching has required linear space. As the pattern grows, your data structure grows at the same rate.

Intuitively, this makes sense. You want to check the text completely matches the pattern, so surely you need to store the whole pattern in order to do that?

However, Porat and Porat (2009) proved that with some chance of error, you can achieve pattern matching in logarithmic space, becoming the first to break this barrier.
\end{frame}

%------------------------------------------------

\begin{frame}
\frametitle{Other accomplishments}
Breslauer and Galil (2011) improved Porat and Porat's complexity from $O(n\log m)$ to $O(n)$ time.

Jalsenius, Porat and Sach (2012) created a solution to parameterised matching in sublinear space and $O(n)$ time.

Clifford et al. (to be published) created a solution to pattern matching with multiple patterns in sublinear space and $O(n\log m)$ time.
\end{frame}

%------------------------------------------------
\section{Looking ahead}
%------------------------------------------------

\begin{frame}
\frametitle{Some theory problems}
\begin{itemize}
    \item\textbf{Parameterised matching:} Current space lower bound is linear in the size of the alphabet. Can we bridge this gap?
    \item\textbf{Dictionary matching:} Can we improve the solution to $O(n)$ time?
    \item\textbf{Parameterised dictionary matching:} Currently no sublinear solution known.
    \item\textbf{Pattern matching against many texts:} Currently no sublinear solution known.
    \item\textbf{k-mismatch:} Currently no sublinear solution known.
    \item\textbf{Distance Matching:} Currently no sublinear solution known.
\end{itemize}
\end{frame}

%------------------------------------------------

\begin{frame}
\frametitle{Pattern matching under annotated streams}
Annotated data streaming is a recent model of computation proposed by Chakrabarti et al. (2012).

Alongside the data stream, the computer also has access to a second stream of extra information from a more powerful computer.

There are very few pattern matching solutions under this model right now.
\end{frame}

%------------------------------------------------

\begin{frame}
\frametitle{Pratical problems}
Very little research at practical performance of algorithms.

How can we make these algorithms run better in practice?

Can we incorporate practical concepts such as parallelism or vectorisation?
\end{frame}

%------------------------------------------------

\begin{frame}
\frametitle{Why are we suitable for this project?}
I developed the first implementation of Breslauer and Galil's algorithm for exact pattern matching, which yielded positive results:

\begin{center}
\includegraphics[width=0.5\paperwidth]{exact_size}
\end{center}

I have also spent this past six months working on the first implementation of Clifford et al.'s algorithm for dictionary matching.
\end{frame}

%------------------------------------------------

\begin{frame}
\frametitle{Work packages}
\begin{itemize}
\item\textbf{WP1:} (Years 1-3) Develop and improve algorithms for pattern matching problems in suiblinear space under the classic streaming model.
\item\textbf{WP2:} (Years 1-3) Develop new algorithms for pattern matching utilising annotated data streams.
\item\textbf{WP3:} (Years 3-5) Implement new and recent algorithms to see how they perform when compared to classic methods on real data.
\end{itemize}
\end{frame}

%------------------------------------------------
\section{The End}
%------------------------------------------------

\begin{frame}
\Huge{\centerline{Questions?}}
\end{frame}

%----------------------------------------------------------------------------------------

\end{document} 
